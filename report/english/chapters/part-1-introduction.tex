% What is the motivation for your research?
% What problem are you trying to solve with your application?
% What are your research questions?
% What are the main contributions of your thesis?

\makeatletter
\@addtoreset{section}{part}
\makeatother



\section{Background}
The rise of information technology has significantly impacted business operations, with software applications becoming integral to sales operations. However, choosing the right software and platform can be challenging, especially for small and medium-sized enterprises (SMEs).

Traditionally, many businesses have used monolithic architecture, a unified and self-contained model. While this model is convenient in the early stages of a project, it can become complex and hard to manage as the application grows.

To address these challenges, businesses are now turning to the MEVN stack (MongoDB, Express.js, Vue.js, Node.js). This stack allows for a scalable and maintainable system, providing a rich, user-friendly interface.

For instance, a sports equipment shop website can be developed using the MEVN stack. By breaking down the application into smaller parts, it's easier to scale services based on demand and deploy updates more frequently without disrupting the entire system.

In the case of our sports equipment shop website, we will be using the MEVN stack to manage products, orders, and payments. Inventory management wil not be included in the initial version of the system. Instead, it's planned to be incorporated in a future update. This approach allows for the system to be flexible and scalable, accommodating new features as needed while ensuring system reliability.
\section{Problem statement}

SMEs face difficulties in selecting suitable software and platforms for their evolving digital operations. Traditional monolithic architecture, while initially convenient, becomes cumbersome and challenging to manage as applications grow, hindering scalability and user experience.

For online sports equipment shops, managing products, orders, and payments is vital. However, due to constraints, the initial system version will be include inventory management, but it is planned for a future update. This strategy ensures flexibility, scalability, and system reliability, addressing key challenges

The core problem lies in developing a user-friendly, scalable, and reliable online sports equipment shop using a technology stack that overcomes the limitations of traditional monolithic architecture and effectively manages the fundamental requirements of an e-commerce website.


% What are your research questions?
\section{Research Objectives}
The primary aim of this project is to design and implement an e-commerce application for a sports equipment shop, along with a front-end application using Vue.js. The specific objectives are as follows:
\begin{itemize}
    \item[-] \textbf{Literature Review:} Conduct a comprehensive review of existing literature on the MEVN stack, its benefits, challenges, and best practices. This will provide a theoretical foundation for the project.
    \item[-] \textbf{Design:} Design a MEVN-based e-commerce application model. The design should consider factors such as scalability, fault tolerance, and ease of adding new features.
    \item[-] \textbf{Development:} Develop a minimum viable product (MVP) of an e-commerce application for a sports equipment shop website using the proposed model. The MVP will include key features such as product management, order processing, and payment processing. Inventory management will be considered for future updates.
    \item[-] \textbf{Documentation:} DDocument the entire process, including the design decisions made, challenges encountered, solutions implemented, and lessons learned. This will serve as a valuable resource for future projects of similar nature.
\end{itemize}
% What are the main contributions of your thesis?
\section{Research Scope}
The focus of this research is on the design and implementation of a MEVN-based e-commerce application for a sport equipment shop website.
The specific areas covered in this research include:
\begin{itemize}
    \item[-] \textbf{MEVN Stack:} The research will delve into the principles and practices of the MEVN stack. It will explore how to design and implement an e-commerce application using this stack.
    \item[-] \textbf{ Front-end Compatibility: } The research will explore the compatibility of the developed back-end API with various front-end frameworks. While the API is designed to be standalone and can work with any front-end framework, the effectiveness of this design will be evaluated
    \item[-] \textbf{Front-end Development with Vue.js:}The research will delve into the integration of the developed back-end API with Vue.js. While the API is designed to be standalone and compatible with any front-end framework, Vue.js has been chosen for its approachability and popularity.
    \item[-] \textbf{Key Features: } The research will focus on implementing key features for an e-commerce application, such as product management, inventory management, order processing, and payment processing.
    \item[-] \textbf{Evaluation:} The research will include an evaluation of the implemented system in terms of functionality, performance, scalability, and reliability.
\end{itemize}

Please note that while the research aims to cover these areas, it is limited by the project's short time. Therefore, the e-shop application developed as part of this project will be a minimum viable product (MVP) with basic features.
\section{Solution approach}
The approach to solving the problem involves several steps, each designed to ensure the successful implementation of an e-commerce application for a sports equipment shop, along with a front-end application using Vue.js. The steps are as follows:
\begin{enumerate}
    \item[-] \textbf{Literature Review:} The first step involves conducting a comprehensive review of existing literature on the MEVN stack. This will provide a theoretical foundation for the project and inform the design and implementation stages.
    \item[-] \textbf{Design:} The next step is to design the MEVN-based e-commerce application model. The design will take into account factors such as scalability, fault tolerance, and ease of adding new features.
    \item[-] \textbf{Development:} Once the design is complete, the development of the minimum viable product (MVP) begins. This involves coding the back-end services and front-end application, and setting up the necessary databases and interfaces. Inventory management will be considered for future updates.
    \item[-] \textbf{Testing: :} After the MVP is developed, it will be thoroughly tested to ensure it functions as expected. This includes unit testing, integration testing, and system testing.
    \item[-] \textbf{Evaluation:} The MVP will then be evaluated in terms of its functionality, performance, scalability, and reliability. Feedback from this evaluation will be used to identify areas for improvement.
\end{enumerate}

\section{Report structure}
% Trong giai đoạn luận văn đề tài  đã thực hiện được một số công việc liên quan sẽ trình bày trong báo cáo như sau:
% % \setlist[enumerate]{leftmargin=*}
\begin{enumerate}[label=\bfseries Chapter \arabic*.,leftmargin=*]
    \item \textbf{Introduction:} This chapter provides a general overview of the topic, its potential, and practical applications in the future. It introduces the concept of a MEVN-based e-commerce application for a sport equipment shop website.
    \item \textbf{Literature review} This section provides a summary of relevant research on the topic and the foundational knowledge necessary to develop an e-commerce application using the MEVN stack.
    \item \textbf{System design and implementation} This section presents the approach to the problem. It discusses the details of the implementation approach, including the tools used.
    \item \textbf{Testing and evaluation.}  This chapter presents the testing plan and management, testing scenarios for the main functions of the system.
    \item \textbf{Conclusion} This section presents the results achieved, the remaining limitations, and the system’s further development.
\end{enumerate}


