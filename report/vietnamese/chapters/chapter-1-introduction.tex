\section{Giới thiệu đề tài}
\section{Đặt vấn đề}
Theo đó ta có những vẫn đề cần giải quyết như sau:
\begin{itemize}
    \item Làm thế nào để giải quyết bài toán trên?
    \item Cách tiếp cận như thế nào?
    \item Những công nghệ nào đã và hiện đang được sử dụng?
    \item Áp dụng nó như thế nào vào bài toán?
    \item Có thể xây dựng trong khoản thời gian hạn hẹp?
\end{itemize}

\section{Mục tiêu đề tài}
Mục tiêu đề tài là nghiêm cứu và xây dựng mô hình ứng dụng để xây dựng ưng dụng thương mại hiệu quả. Từ đó, ta có thể xây dựng giải pháp và cũng như sản phẩm cuối là Ứng dụng thương mại cho một cửa hàng bán hàng lưu niệm.

\section{Giới hạn đề tài}
Dựa theo mục tiêu đề và thời gian thực hiện, ta sẽ giới hạn ở phạm vi tìm hiểu và áp dụng bằng việc xây dựng ứng dụng với các chức năng nghiệm vụ tối thiểu cần có cho ứng dụng bán hàng lưu niệm.

\section{Tiếp cận giải pháp}
Như vậy để thực hiện theo đúng mục tiêu của đề tài cần xác định một số công việc phải giải
quyết như sau:
\begin{itemize}
    \item Tìm hiểu các phương pháp tiếp cận đã được hiện thực.
    \item Tìm hiểu các giải pháp hay mô hình phù hợp với nội dung đề tài.
    \item Lựa chọn mô hình và công nghệ tương ứng với đề tài.
    \item Lên kế hoặch thực hiện và pháp triển giải pháp ứng dụng và kiểm thử.
\end{itemize}


\section{Cấu trúc báo cáo}
Trong giai đoạn luận văn đề tài nhóm đã thực hiện được một số công việc liên quan sẽ trình bày trong báo cáo như sau:

\begin{itemize}
    \item Chương 1: Giới thiệu tổng quan về nhận diện hướng nhìn, cũng là chương hiện hành. Trong chương này sẽ đưa đến cái nhìn tổng quát về đề tài, tiềm năng và ứng dụng thực tế trong tương lai.
    \item Chương 2: Tổng quan một số công trình nghiên cứu liên quan tới đề tài mà nhóm tìm hiểu được qua ba phần: hướng tiếp cận, mô hình sử dụng và kết quả đạt được.
    \item Chương 3: Các kiến thức nền tảng phục vụ việc pháp triển mô hình microservices sử dủng ASP.NET và xây dựng ứng dụng MVP sử dụng mô hình Clean Achritecture trong Flutter.
    \item Chương 4: Trình bày hướng tiếp cận bài toán: chương này sẽ đi vào chi tiết cách tiếp cận thực hiện mô hình, bao gồm công cụ sử dụng.
    \item Chương 5: Tổng kết những công việc nhóm đã làm được, đánh giá và định hướng kế hoạch mà nhóm tiếp tục phát triển trong niên luận.
\end{itemize}


